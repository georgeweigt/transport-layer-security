\documentclass[11pt]{article}
\usepackage{fancyhdr,amsmath,amsfonts,mathrsfs,graphicx,parskip,listings}

% change margins
\addtolength{\oddsidemargin}{-.875in}
\addtolength{\evensidemargin}{-.875in}
\addtolength{\textwidth}{1.75in}
\addtolength{\topmargin}{-.875in}
\addtolength{\textheight}{1.75in}

\begin{document}

\begin{center}
{\sc SSL NOTES}
\end{center}

\subsection*{Record Format}
SSL runs on top of TCP.
Recall that TCP is a stream protocol that erases packet boundaries.
In order to provide for in-band signalling in a stream protocol,
SSL partitions the stream into records.
Each record has a five byte header followed by a payload as shown
in the following diagram.

\begin{center}
\begin{tabular}{|ll|}
\hline
Content Type & 1 byte\\
\hline
Version & 2 bytes\\
\hline
Length $n$ & 2 bytes\\
\hline
&\\
&\\
Payload & $n$ bytes\\
&\\
&\\
\hline
\end{tabular}
\end{center}

Synchronization is maintained by the length field which indicates not only
the payload length but also the start of the next record.
When SSL switches to encrypted mode, only the payload is encrypted.
The header is always sent in the clear to maintain synchronization.

RFC 2246 defines the following content types. (See RFC 2246 page 17.)

\begin{center}
\begin{tabular}{|l|c|}
\hline
Content Type & Decimal Code\\
\hline
Change Cipher Spec & 20\\
Alert Message & 21\\
Handshake Protocol & 22\\
Application Data & 23\\
\hline
\end{tabular}
\end{center}

\subsection*{Handshake Protocol}
SSL defines a handshake protocol for sending connection setup messages.
The handshake protocol uses content type 22 (0x16) in the
first byte of the record header.
Handshake messages have no alignment relative to the record protocol
and may be split across non-contiguous records.
The handshake header and data are shown in the following diagram.

\begin{center}
\begin{tabular}{|ll|}
\hline
Handshake Type & 1 byte\\
\hline
Handshake Length $m$ & 3 bytes\\
\hline
&\\
&\\
Handshake Payload & $m$ bytes\\
&\\
&\\
\hline
\end{tabular}
\end{center}

\subsection*{Typical HTTPS Transaction}
The following table shows what happens during a
typical HTTPS session.
Messages in angle brackets are encrypted.

\begin{center}
\begin{tabular}{|ccc|}
\hline
Client & & Server\\
\hline
& & \\
Client Hello & $\longrightarrow$ &\\
& $\longleftarrow$ & Server Hello\\
& $\longleftarrow$ & Certificate\\
& $\longleftarrow$ & Server Hello Done\\
Client Key Exchange & $\longrightarrow$ &\\
Change Cipher Spec & $\longrightarrow$ &\\
$\langle$ Finished $\rangle$ & $\longrightarrow$ &\\
& $\longleftarrow$ & Change Cipher Spec\\
& $\longleftarrow$ & $\langle$ Finished $\rangle$\\
& &\\
$\langle$ HTTP Get $\rangle$ & $\longrightarrow$ & \\
& $\longleftarrow$ & $\langle$ HTTP Response $\rangle$\\
& & \\
& $\longleftarrow$ & $\langle$ Close Notify $\rangle$\\
$\langle$ Close Notify $\rangle$ & $\longrightarrow$ &\\
\hline
\end{tabular}
\end{center}

\subsection*{Certificates}
A certificate is a collection of type-length-value objects (TLVs) with the
property that TLVs can be nested.
In other words, the V of a TLV can itself contain more TLVs.
Nested TLVs are indicated by a type field T of either 0x30 (SEQUENCE)
or 0x31 (SET).
Certificates use nested TLVs to organize certificate values
into groups.
The following diagram is a general outline of how certificate data is organized
(not all TLVs are shown).

\begin{lstlisting}
  SEQUENCE
  | SEQUENCE                Certificate Info
  | | INTEGER               Serial Number
  | | SEQUENCE              Certificate Signature Algorithm
  | | SEQUENCE              Issuer
  | | SEQUENCE              Validity
  | | SEQUENCE              Subject
  | | SEQUENCE              Subject Public Key Info
  | | | SEQUENCE
  | | | | OBJECT ID         Subject Public Key Algorithm
  | | | | NULL
  | | | BIT STRING          Subject Public Key
  | | | | SEQUENCE
  | | | | | INTEGER         Modulus (n in RSA encryption}
  | | | | | INTEGER         Exponent (e in RSA encryption
  | SEQUENCE
  | | OBJECT ID             Certificate Signature Algorithm
  | | NULL
  | BIT STRING              Certificate Signature Value
\end{lstlisting}

The client computer uses the certificate key to encrypt its pre-master
secret.
The encrypted pre-master secret is then sent to the server.
The certificate key is not used for any other type of encryption.
After the connection is established, application data is encrypted using
keys derived from the pre-master secret.

Self-signed certificates can be checked by encrypting the signature value
using the public key found in the certificate itself.
The result is then compared to a hash of the Certificate Info data.

\subsection*{RSA encryption}
The client computer
uses RSA encryption to securely transmit the client's
48-byte pre-master secret to the web server.

Let $n$ and $e$ be RSA public keys where $n$ is the modulus and $e$
is the exponent.
Typically $e=65,537$ and $n$ is either 1,024 or 2,048 bits long.
Let $P$ be a plaintext such that bit-wise numerically $P<n$.
Then ciphertext $C$ is obtained from $P$ by raising $P$ to the
power $e$ modulo $n$.
$$C=P^e\bmod n$$
Only the originator of the public keys $n$ and $e$ can decrypt
$C$ and obtain $P$.
The function ``modpow'' in ssl-demo.c implements $P^e\bmod n$.

There is also a padding requirement that ensures a strong cipher.
The padding is specified in RFC 3447 page 25 as follows.
$$P=\text{0x00}\mid\text{0x02}\mid U\mid\text{0x00}\mid M$$
$M$ is the actual message to encrypt.
$U$ is a sequence of non-zero random bytes.
The vertical bar is a concatenation operator.
The length of $U$ is chosen such that the bit length of $P$ is the
same as $n$.
The number of bytes in $U$ cannot be less than eight.
If $n$ is 1,024 bits long (128 bytes) then $M$ cannot be longer than
$128-11=117$ bytes.
Note that the leading 0x00 ensures that $P<n$ since $n$ must have
its most significant bit set.

\section*{References}

A Layman's Guide to a Subset of ASN.1, BER, and DER\newline
http://luca.ntop.org/Teaching/Appunti/asn1.html

FIPS Publication 180-4, Secure Hash Standard

RFC 1321 The MD5 Message-Digest Algorithm

RFC 2104 HMAC: Keyed-Hashing for Message Authentication

RFC 2246 The TLS Protocol Version 1.0

RFC 3447 Public-Key Cryptography Standards (PKCS) \#1: RSA Cryptography
Specifications Version 2.1

\end{document}
